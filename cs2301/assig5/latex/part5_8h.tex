\section{part5.h File Reference}
\label{part5_8h}\index{part5.h@{part5.h}}
\subsection*{Functions}
\begin{CompactItemize}
\item 
int \bf{read\-Employee\-Log} (int index, char $\ast$name, int $\ast$salary)
\item 
int \bf{num\-Employees} ()
\end{CompactItemize}


\subsection{Function Documentation}
\index{part5.h@{part5.h}!numEmployees@{numEmployees}}
\index{numEmployees@{numEmployees}!part5.h@{part5.h}}
\subsubsection{\setlength{\rightskip}{0pt plus 5cm}int num\-Employees ()}\label{part5_8h_801748b5d2d028f0482ce56bb43d331c}


Calculates number of employees in text file \begin{Desc}
\item[Returns:]number of employees \end{Desc}
\begin{Desc}
\item[Note:]still uses fgets since that was only way to count lines \end{Desc}
\index{part5.h@{part5.h}!readEmployeeLog@{readEmployeeLog}}
\index{readEmployeeLog@{readEmployeeLog}!part5.h@{part5.h}}
\subsubsection{\setlength{\rightskip}{0pt plus 5cm}int read\-Employee\-Log (int {\em index}, char $\ast$ {\em name}, int $\ast$ {\em salary})}\label{part5_8h_6eba06bb925710329353f8f4bcfa3192}


Reads 1 employee from a text file \begin{Desc}
\item[Parameters:]
\begin{description}
\item[{\em index}]of the employee number to retrieve \item[{\em name}]of employer to save to \item[{\em salary}]pointer to save information to \end{description}
\end{Desc}
\begin{Desc}
\item[Returns:]int 1 if failed to read file, 0 if succeeded \end{Desc}
