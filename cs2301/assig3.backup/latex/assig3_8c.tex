\section{assig3.c File Reference}
\label{assig3_8c}\index{assig3.c@{assig3.c}}
{\tt \#include $<$stdio.h$>$}\par
{\tt \#include $<$stdlib.h$>$}\par
\subsection*{Defines}
\begin{CompactItemize}
\item 
\#define \bf{MAX\_\-NUMS}~(10)
\end{CompactItemize}
\subsection*{Functions}
\begin{CompactItemize}
\item 
int \bf{fill\_\-array} (const char $\ast$argv[$\,$], int $\ast$new\_\-array, int max\_\-elems, int argc)
\item 
void \bf{print\_\-array} (int $\ast$array, int num\_\-elements)
\item 
void \bf{sort\_\-array} (int $\ast$array, int num\_\-elements)
\item 
int \bf{main} (int argc, const char $\ast$argv[$\,$])
\begin{CompactList}\small\item\em main function \item\end{CompactList}\end{CompactItemize}


\subsection{Define Documentation}
\index{assig3.c@{assig3.c}!MAX_NUMS@{MAX\_\-NUMS}}
\index{MAX_NUMS@{MAX\_\-NUMS}!assig3.c@{assig3.c}}
\subsubsection{\setlength{\rightskip}{0pt plus 5cm}\#define MAX\_\-NUMS~(10)}\label{assig3_8c_c7aa44a8ba8ea1f8b65b9b91481120cc}




\subsection{Function Documentation}
\index{assig3.c@{assig3.c}!fill_array@{fill\_\-array}}
\index{fill_array@{fill\_\-array}!assig3.c@{assig3.c}}
\subsubsection{\setlength{\rightskip}{0pt plus 5cm}int fill\_\-array (const char $\ast$ {\em argv}[$\,$], int $\ast$ {\em new\_\-array}, int {\em max\_\-elems}, int {\em argc})}\label{assig3_8c_7efe7f2b902b9a5c80f320b570621e5a}


This function fills the elements of an array with that of another \begin{Desc}
\item[Parameters:]
\begin{description}
\item[{\em argv}]const char pointer to argv array from command line \item[{\em new\_\-array}]pointer to first element in array to be filled \item[{\em max\_\-elems}]number of maximum elements that can be placed in array \item[{\em argc}]number of arguments on command line \end{description}
\end{Desc}
\begin{Desc}
\item[Returns:]0 for failure, otherwise num of elements in array \end{Desc}
\index{assig3.c@{assig3.c}!main@{main}}
\index{main@{main}!assig3.c@{assig3.c}}
\subsubsection{\setlength{\rightskip}{0pt plus 5cm}int main (int {\em argc}, const char $\ast$ {\em argv}[$\,$])}\label{assig3_8c_c0f2228420376f4db7e1274f2b41667c}


main function 

\index{assig3.c@{assig3.c}!print_array@{print\_\-array}}
\index{print_array@{print\_\-array}!assig3.c@{assig3.c}}
\subsubsection{\setlength{\rightskip}{0pt plus 5cm}void print\_\-array (int $\ast$ {\em array}, int {\em num\_\-elements})}\label{assig3_8c_8d05914dfc431857377538709fc5d8a2}


This function prints the elements of an array, one per line. This is modified from the example in Ciaraldi's Class 7 Example on slide 36 \begin{Desc}
\item[Parameters:]
\begin{description}
\item[{\em array}]pointer to first element in array (base) \item[{\em num\_\-elements}]number of elements actually filled in the array \end{description}
\end{Desc}
\begin{Desc}
\item[Returns:]Nothing \end{Desc}
\begin{Desc}
\item[Note:]the highest index should be num\_\-elements-1 \end{Desc}
\index{assig3.c@{assig3.c}!sort_array@{sort\_\-array}}
\index{sort_array@{sort\_\-array}!assig3.c@{assig3.c}}
\subsubsection{\setlength{\rightskip}{0pt plus 5cm}void sort\_\-array (int $\ast$ {\em array}, int {\em num\_\-elements})}\label{assig3_8c_8896f0ce55dd25259e2141e714d6ef93}


This function sorts the elements of an array of ints from greatest to least This is modified from the example in Ciaraldi's Class 7 Example on slide 28 \begin{Desc}
\item[Parameters:]
\begin{description}
\item[{\em array}]pointer to first element in array (base) \item[{\em num\_\-elements}]number of elements actually filled in the array \end{description}
\end{Desc}
\begin{Desc}
\item[Returns:]Nothing \end{Desc}
\begin{Desc}
\item[Note:]uses bubble sort algorithm developed by tcpong @ cs.ust.hk \end{Desc}
