\section{sort.c File Reference}
\label{sort_8c}\index{sort.c@{sort.c}}
File for sorting functions for assignment 3. 

{\tt \#include $<$stdio.h$>$}\par
{\tt \#include $<$stdlib.h$>$}\par
{\tt \#include \char`\"{}sort.h\char`\"{}}\par
\subsection*{Functions}
\begin{CompactItemize}
\item 
int $\ast$ \bf{allocate\_\-array} (const char $\ast$argv[$\,$], int argc, int $\ast$num\_\-elems)
\item 
void \bf{fill\_\-array} (const char $\ast$argv[$\,$], int $\ast$new\_\-array, int $\ast$num\_\-elems)
\item 
void \bf{print\_\-array} (int $\ast$array, int num\_\-elements)
\item 
void \bf{sort\_\-array} (int $\ast$array, int num\_\-elements)
\end{CompactItemize}


\subsection{Detailed Description}
File for sorting functions for assignment 3. 

\begin{Desc}
\item[Author:]Christopher Bove (cpbove) \end{Desc}
\begin{Desc}
\item[Date:]6 April 2015 \end{Desc}


\subsection{Function Documentation}
\index{sort.c@{sort.c}!allocate_array@{allocate\_\-array}}
\index{allocate_array@{allocate\_\-array}!sort.c@{sort.c}}
\subsubsection{\setlength{\rightskip}{0pt plus 5cm}int$\ast$ allocate\_\-array (const char $\ast$ {\em argv}[$\,$], int {\em argc}, int $\ast$ {\em num\_\-elems})}\label{sort_8c_305a5d36cdfe1ffeff2da8f46d3d5454}


This function allocates an array of a sufficient size to store the numbers passed in argv[]. \begin{Desc}
\item[Parameters:]
\begin{description}
\item[{\em argv}]const char pointer to argv array from command line \item[{\em argc}]number of arguments on command line \end{description}
\end{Desc}
\begin{Desc}
\item[Returns:]pointer to base of array of numbers \end{Desc}
\begin{Desc}
\item[Note:]This does not check type of string on line. It is assumed to be int. \end{Desc}
\index{sort.c@{sort.c}!fill_array@{fill\_\-array}}
\index{fill_array@{fill\_\-array}!sort.c@{sort.c}}
\subsubsection{\setlength{\rightskip}{0pt plus 5cm}void fill\_\-array (const char $\ast$ {\em argv}[$\,$], int $\ast$ {\em new\_\-array}, int $\ast$ {\em num\_\-elems})}\label{sort_8c_710b7ff37ab6f6ab4ccce28c0f6b8c48}


This function fills the elements of an array with that of argv \begin{Desc}
\item[Parameters:]
\begin{description}
\item[{\em argv}]const char pointer to argv array from command line \item[{\em new\_\-array}]pointer to first element in array to be filled \item[{\em max\_\-elems}]number of maximum elements that can be placed in array \item[{\em argc}]number of arguments on command line \end{description}
\end{Desc}
\index{sort.c@{sort.c}!print_array@{print\_\-array}}
\index{print_array@{print\_\-array}!sort.c@{sort.c}}
\subsubsection{\setlength{\rightskip}{0pt plus 5cm}void print\_\-array (int $\ast$ {\em array}, int {\em num\_\-elements})}\label{sort_8c_8d05914dfc431857377538709fc5d8a2}


This function prints the elements of an array, one per line. This is modified from the example in Ciaraldi's Class 7 Example on slide 36 \begin{Desc}
\item[Parameters:]
\begin{description}
\item[{\em array}]pointer to first element in array (base) \item[{\em num\_\-elements}]number of elements actually filled in the array \end{description}
\end{Desc}
\begin{Desc}
\item[Note:]the highest index should be num\_\-elements-1 \end{Desc}
\index{sort.c@{sort.c}!sort_array@{sort\_\-array}}
\index{sort_array@{sort\_\-array}!sort.c@{sort.c}}
\subsubsection{\setlength{\rightskip}{0pt plus 5cm}void sort\_\-array (int $\ast$ {\em array}, int {\em num\_\-elements})}\label{sort_8c_8896f0ce55dd25259e2141e714d6ef93}


This function sorts the elements of an array of ints from greatest to least This is modified from the example in Ciaraldi's Class 7 Example on slide 28 \begin{Desc}
\item[Parameters:]
\begin{description}
\item[{\em array}]pointer to first element in array (base) \item[{\em num\_\-elements}]number of elements actually filled in the array \end{description}
\end{Desc}
\begin{Desc}
\item[Note:]uses bubble sort algorithm developed by tcpong @ cs.ust.hk \end{Desc}
